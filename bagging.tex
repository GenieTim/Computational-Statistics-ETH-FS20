\section*{Bootstrap Aggregating (Bagging)}
Prevent overfitting, reduce variance.

\textbf{Bagging for Regression:} For data $(X_1,Y_1),...,(X_n,Y_n)$ and base procedure $\hat g(\cdot): \mathbb{R}^p\to \mathbb{R}$, take $B$ bootstrap samples $\hat g_{bag}(x) = \tfrac 1 B \sum_{b=1}^B \hat g^{*b}(x)$ where $\hat g^{*b}$ is the estimate based on the $b$-th bootstrap sample. No pruning, since variance of single tree not a problem as we average. Linear predictions are the same under bagging, so only interesting for non-linear estimates. For regression can only improve or stay the same.

\textbf{Bagging for Classification:}
$\hat g(\cdot): \mathbb{R}^p \to \{1, ..., k\}$. $\hat g (x) = \text{argmax}_{k=1,...,K} \sum_{b=1}^B \mathds{1}_{\hat g^{*b}(x)=k}$ (majority vote).
Can also get class probability: $\hat p_k^{bag}(x) = \tfrac{1}{B} \sum_{b=1}^B \hat p_k^{*b}(x)$.
Can also be formulated as $\hat g^{bag}(x) = \text{argmax}_{k=1,...,K} \hat p_k^{bag}(x)$ (better if interested in class probabilities, sometimes even helps accuracy).
Has the advantages that we get the estimated class probabilities; bagging a good classifier can improve performance (bad classifier can decrease).

\textbf{Random Forests:}
Essentially bagged trees. Have $B$ bootstrap samples $\to$ create trees.
They reduce dependence between tree estimates by only allowing a random subset of predictors at each split.
Default: regression $p/3$, classification $\sqrt{p}$ (in R option mtry).

\textbf{Out-of-Bag Error:}
Some bags have not trained on a particular sample.
Can predict this only by the bags that have not been trained on it (should be $\sim 1/3$) for all samples and average to get a valid estimate for the test error.

\begin{tabular}{llll}
                   & Tree & Bagging & Random Forest \\
  Performance      & -    & +       & ++            \\
  Computation      & +    & -       & +/-           \\
  Interpretation   & +    & -       & -             \\
  Out-of-bag error & -    & +       & +
\end{tabular}%

\begin{codebox}{r}{Random Forests}
  library(randomForest)
  cs.bag <- randomForest(Sales~., train.data, mtry=p-1) # Bagging
  cs.forest <- randomForest(Sales~., train.data, mtry=p/3) # Random Forest
  importance(cs.forest) # Importance of predictors. Plot: varImpPlot(...
\end{codebox}
