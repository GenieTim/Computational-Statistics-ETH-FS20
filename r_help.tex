\section*{General R commands}

Packages used: boot, leaps, gam, glmnet, ISLR, rpart, tree, randomForest, kknn, class, splines, MASS, Datasets

\begin{codebox}{r}{R-Help}
  is.na(x) # returns Boolean vector of size length(x)
  na.omit(x) # removes rows with missing values from dataset.
  qt(), rt(), qf(), rf() # etc. for student/r distr.
  cut(seq(1,n),breaks=K,labels=FALSE) # to get K (roughly) equally sized folds.
  which(x, arr.ind = TRUE) # gives vector of indices for which x==TRUE
  par(mfrow=c(1,1)) # for division of plot window.
  unique(x) # vector with elements of x # without duplicates.
  names(object) # gives the stuff that can be returned by object$...
  abline(v=...) # for vertical line in plot (h for horizontal)
  cbind(), rbind() # combine cols resp. rows
  hist(..., freq=F) # for probab. scale
  levels() # access to the levels attribute of a variable
  complete.cases(data) # removes NA's
  stripchart(..., method="stack") # for small data sets
  fitdistr(data, densfun,...) # fits MLE do data (e.g. densfun="'gamma"')
  which.max(...) # returns indices of maxima
  # Test if an element is in a list
  if ("X1" %in% names(coef(m,mo)))
  # Creating categorical variables
  High=ifelse(Carseats$Sales<=8,"No","Yes")
  # Standardize data
  scaled.dat <- scale(dat)
  # Anova test to determine if there is a significant
  # difference between models. Anova uses RSS and DoF
  # of largest (last) model, so use ascending order!
  anova(fit.0, fit.1, fit.2, fit.3)
  # Given fixed x, error distribution and true param.:
  # Power of test simulation. Know that y ~ poly(x, 3) + err (for typeI error: do same with y = err)
  results.power <- numeric(n.sim)
  for (i in 1:n.sim) {
      err <- rgamma(n, ...) - 2
      y <- beta.0 + beta.1 * I(x) + beta.2 * I(x^2) + beta.3 + I(x^3) + err
      fit.power <- lm(y ~ I(x) + I(x^2) + I(x^3))
      f1 <- summary(fit.power)$fstatistic
      p.val.power <- 1 - pf(f1[1], f1[2], f1[3])
      results.power[i] <- p.val.power < 0.05
    }
  power <- mean(results.power)
  # draw density and CDF
  grid <- seq(from=0,to=5,length=200)
  plot(grid, dlnorm(grid), type="l", main="density")
  plot(grid, plnorm(grid), type="l", main="CDF")
\end{codebox}

